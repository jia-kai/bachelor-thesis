% $File: main.tex
% $Date: Fri Jun 05 18:20:38 2015 +0800
% $Author: jiakai <jia.kai66@gmail.com>

\PassOptionsToPackage{bookmarksdepth=3}{hyperref}
\documentclass[bachelor]{thuthesis}
%\documentclass[master]{thuthesis}
%\documentclass[doctor]{thuthesis}
% \documentclass[%
%   bachelor|master|doctor|postdoctor, % mandatory option
%   secret,
%   openany|openright,
%   arialtoc,arialtitle]{thuthesis}

% 所有其它可能用到的包都统一放到这里了,可以根据自己的实际添加或者删除。
\usepackage{thutils}
\usepackage{pdfpages}
\usepackage{paralist}
% $File: mathdef.tex
% $Date: Fri Jun 05 18:42:54 2015 +0800
% $Author: jiakai <jia.kai66@gmail.com>

\usepackage{bm}
\usepackage{mathtools}

\newcommand{\trans}[1]{#1^\intercal}
\numberwithin{equation}{section}
\renewcommand{\vec}[1]{\boldsymbol{#1}}
\newcommand{\tvec}[1]{\tilde{\vec{#1}}}
\newcommand{\dif}{\mathrm{d}}
\newcommand{\prob}[2][]{p_{#1}\left(#2\right)}
\renewcommand{\det}[1]{\text{det}\left(#1\right)}
\newcommand{\cov}[1]{\text{cov}\left(#1\right)}
\newcommand{\corr}[1]{\text{corr}\left(#1\right)}
\newcommand{\abs}[1]{\left|#1\right|}
\newcommand{\real}{\mathbb{R}}

\DeclarePairedDelimiter{\ceil}{\lceil}{\rceil}
\DeclarePairedDelimiter{\floor}{\lfloor}{\rfloor}
\DeclareMathOperator*{\argmin}{arg\,min}
\DeclareMathOperator*{\argmax}{arg\,max}

% for section cnn
\newcommand{\augdataset}[1]{D^{\mathcal{T}}(\vec{#1})}
\newcommand{\augftrset}[1]{S_{\vec{W}}^{\mathcal{T}}(\vec{#1})}

% vim: filetype=tex foldmethod=marker foldmarker=f{{{,f}}}


% custom commands

\setcounter{tocdepth}{2}


% \textref{marker}{text}
\newcommand{\textref}[2]{\hyperref[#1]{#2}}
\newcommand{\figref}[1]{\hyperref[fig:#1]{\figurename~\ref*{fig:#1}}}
\newcommand{\tabref}[1]{\hyperref[tab:#1]{\tablename~\ref*{tab:#1}}}
\newcommand{\secref}[1]{\hyperref[sec:#1]{\ref*{sec:#1}节}}
\newcommand{\eqnref}[1]{\hyperref[eqn:#1]{(\ref*{eqn:#1})}}

\newcommand{\addplot}[1]{\centering
    \includegraphics[width=0.9\textwidth,
    height=0.4\paperheight,keepaspectratio]{#1}}


% 你可以在这里修改配置文件中的定义,导言区可以使用中文。
\def\myname{贾开}

\begin{document}


%%% 封面部分
\frontmatter

%%% Local Variables:
%%% mode: latex
%%% TeX-master: t
%%% End:
\secretlevel{绝密} \secretyear{2100}

\ctitle{基于非监督深度学习的医学影像特征提取研究}
% 根据自己的情况选,不用这样复杂
\makeatletter
\ifthu@bachelor\relax\else
  \ifthu@doctor
    \cdegree{工学博士}
  \else
    \ifthu@master
      \cdegree{工学硕士}
    \fi
  \fi
\fi
\makeatother


\cdepartment[计算机]{计算机科学与技术系}
\cmajor{计算机科学与技术}
\cauthor{贾开}
\csupervisor{宋亦旭~~副研究员}
% 如果没有副指导老师或者联合指导老师,把下面两行相应的删除即可。
%\cassosupervisor{陈文光教授}
%\ccosupervisor{某某某教授}
% 日期自动生成,如果你要自己写就改这个cdate
%\cdate{\CJKdigits{\the\year}年\CJKnumber{\the\month}月}

% 博士后部分
% \cfirstdiscipline{计算机科学与技术}
% \cseconddiscipline{系统结构}
% \postdoctordate{2009年7月——2011年7月}

\etitle{Research on Unsupervised Deep Feature Learning on Medical Images}
% 这块比较复杂,需要分情况讨论:
% 1. 学术型硕士
%    \edegree:必须为Master of Arts或Master of Science(注意大小写)
%              “哲学、文学、历史学、法学、教育学、艺术学门类,公共管理学科
%               填写Master of Arts,其它填写Master of Science”
%    \emajor:“获得一级学科授权的学科填写一级学科名称,其它填写二级学科名称”
% 2. 专业型硕士
%    \edegree:“填写专业学位英文名称全称”
%    \emajor:“工程硕士填写工程领域,其它专业学位不填写此项”
% 3. 学术型博士
%    \edegree:Doctor of Philosophy(注意大小写)
%    \emajor:“获得一级学科授权的学科填写一级学科名称,其它填写二级学科名称”
% 4. 专业型博士
%    \edegree:“填写专业学位英文名称全称”
%    \emajor:不填写此项
\edegree{Doctor of Engineering}
\emajor{Computer Science and Technology}
\eauthor{Kai Jia}
\esupervisor{Yixu Song}
%\eassosupervisor{Chen Wenguang}
% 这个日期也会自动生成,你要改么?
% \edate{December, 2005}

% 定义中英文摘要和关键字
\begin{cabstract}
    我要毕业了,但还没写摘要
\end{cabstract}

\ckeywords{深度学习, 非监督学习, 肝脏影像分割}

\begin{eabstract}
    hello, world
\end{eabstract}

\ekeywords{deep learning, unsupervised learning, liver image segmentation}

% 设置 PDF 文档的作者、主题等属性
\makeatletter
\thu@setup@pdfinfo
\makeatother
\makecover

% 目录
\tableofcontents


%%% 正文部分
\mainmatter
% $File: intro.tex
% $Date: Fri May 29 01:27:37 2015 +0800
% $Author: jiakai <jia.kai66@gmail.com>

\chapter{引言}
本章主要阐述本文要研究的问题,简述其背景,并概述目前相关领域的研究现状。
最后给出本文的结构安排。

\section{研究背景}
对医学影像的获取、分析和处理在现代医疗工作中占有重要地位。
常用的基于放射的医学成像技术包括计算机体层成像(Computed Tomography, CT)、
磁共振成像(Magnetic Resonance Imaging, MRI)等。在医学影像技术的辅助下,
人们可以非侵入式的直接观察到患者体内的情况,
对疾病的诊治和人体科学的研究都有重大意义。

对医学影像的分析和处理也是计算机科学相关领域长久以来研究的热点。
但由于其本身的难度以及对结果精确度的极高要求,这方面的自动化方法远不成熟,
只能作为人工辅助,临床上也一般由训练有素的医生来完成对影像的分析工作。
一般而言,医学影像的如下特点使得对其的自动化处理较为困难:
\begin{inparaenum}[\itshape 1\upshape)]
    \item 分辨率不高,如CT的图像矩阵往往只有$512\times 512$\cite{medimging2};
    \item 来源单一而封闭,需要大型扫描设备,而且涉及患者隐私;
    \item 由于人体构造、扫描时体位等不同而呈现很大差异性,
        病变部位则更是千差万别;
    \item 维度高,一般为3D体数据,或者带时间信息的4D图像。
\end{inparaenum}

近年来随着互联网的高速发展带来的海量数据,
以及GPU技术的成熟和普及带来的高密度高速计算能力,
使得深度学习逐渐成为一种实用的高性能通用模型,并在图像分类\cite{he2015delving}、
人脸识别\cite{schroff2015facenet}等领域的特定数据集上超过了人类水平。
但目前将深度学习用于医学影像处理的工作并不多。因此,
本文将尝试使用基于非监督学习的深度学习算法,自动从CT扫描数据中提取特征,
并基于本文提出的评测方法测试各种特征的性能。


\section{深度学习}


\section{医学影像处理}
医学影像的特点、研究的意义、难点,深度学习与此的结合。

并简述肝脏分割的现有方法,引出本文以刚脏分割作为 example task。

\section{论文结构}


% vim: filetype=tex foldmethod=marker foldmarker=f{{{,f}}}

% $File: ISA.tex
% $Date: Mon Jun 08 22:59:35 2015 +0800
% $Author: jiakai <jia.kai66@gmail.com>

\chapter{基于层叠卷积ISA的特征提取\label{chap:ISA}}
本章主要介绍独立子空间分析(Independent Subspace Analysis, ISA)方法的基本原理,
及将其多次层叠后构成深度网络进行特征提取的方法,并对该方法进行简单的讨论。

\section{ISA的基本原理}
ISA是对独立成分分析(Independent Component Analysis, ICA)的扩展,
是一种经典的统计学习方法。
在本节中,先对ICA进行介绍,再将其扩展到ISA。关于ISA的更为详细的内容,
可以参考\cite{hyvarinen2009natural}。


\subsection{ICA的基本原理}
% f{{{
ICA是一种生成模型,其出发点是希望从对一个随机变量的一系列观察中,
分析出其背后的独立成分,每个成分有自己的概率分布,从而得出该随机变量的概率分布。

具体而言,对随机变量$\vec{x} = \trans{(x_1, \cdots, x_n)}$,ICA假设
\begin{eqnarray}
    \vec{x} = \vec{A}\vec{s}
    \label{eqn:ica:0}
\end{eqnarray}
其中,$\vec{A}$是一个$n\times m$的矩阵,$\vec{s}=\trans{(s_1,\cdots,s_m)}$
是一个$m$维随机变量,对于$i\neq j$,$s_i$和$s_j$独立。

实际应用时,一般$m \le n$,先对$\vec{x}$进行主成分分析(Principal Component
Analysis, PCA),降维成$m$维随机变量$\vec{z}=\vec{P}(\vec{x}-\bar{\vec{x}})$,
其中$\vec{P}$是$m \times n$的PCA矩阵;
另外\eqnref{ica:0}中的$\vec{x}$对应替换成$\vec{z}$,即
\begin{eqnarray}
    \vec{z} = \vec{B}\vec{s}
    \label{eqn:ica:1}
\end{eqnarray}
其中$\vec{B}$是$n\times n$矩阵,
用于将假设的独立特征空间变换到观察到的随机向量空间。显然\eqnref{ica:1}可逆,
记$\vec{B}^{-1}=\vec{V}$,有
\begin{eqnarray}
    \vec{s} = \vec{V}\vec{z}
    \label{eqn:ica:2}
\end{eqnarray}

在\eqnref{ica:0}中,$\vec{s}$解释了$\vec{x}$的概率分布,
可认为是一种更易处理、更能反应$\vec{x}$本质的特征;
而\eqnref{ica:2}则给出了从$\vec{x}$提取特征$\vec{s}$的方法,
其中$\vec{V}$就是特征检测器。

下面将简述基于最大似然来求解$\vec{V}$的方法。基于独立性的假设,有
\begin{eqnarray}
    \prob{\vec{s}} = \prod_{i=1}^m\prob[i]{s_i}
    \label{eqn:ica:3}
\end{eqnarray}
联合\eqnref{ica:2},于是有
\begin{eqnarray}
    \prob{\vec{z}} &=&  \prob{\vec{V}^{-1}\vec{s}} \nonumber \\
        &=& \abs{\det{\vec{V}}} \prob{\vec{s}} \nonumber \\
        &=& \abs{\det{\vec{V}}}\prod_{i=1}^m\prob[i]{s_i} \nonumber \\
        &=& \abs{\det{\vec{V}}}\prod_{i=1}^m\prob[i]{\trans{\vec{v_i}}\vec{z}}
    \label{eqn:ica:4}
\end{eqnarray}

假设对于$\vec{z}$有$T$个独立观测结果$\vec{z_1}\cdots\vec{z_T}$,
可定义最大似然为优化目标,于是ICA的参数$\vec{V}$估计如下:
\begin{eqnarray}
    \vec{V}^* &=& \argmax_{\vec{V}} L(\vec{V}) \nonumber \\
    &=& \argmax_{\vec{V}} \prod_{t=1}^T \prob{\vec{z_t}|\vec{V}} \nonumber \\
    &=& \argmax_{\vec{V}} \prod_{t=1}^T \left(\abs{\det{\vec{V}}}
            \prod_{i=1}^m\prob[i]{\trans{\vec{v_i}}\vec{z_t}} \right)
    \label{eqn:ica:5}
\end{eqnarray}

另外,由独立性,还应该要求$\vec{s}$的各分量不相关,即$\cov{\vec{s}} = \vec{I}$,
则需要$\vec{V}$满足$\trans{\vec{V}}\vec{V}=\vec{I}$,
于是有$\abs{\det{\vec{V}}}=1$,ICA的求解最终为如下形式:
\begin{eqnarray}
    \vec{V}^* &=& \argmax_{\vec{V}}
            \prod_{t=1}^T \left(
            \prod_{i=1}^m\prob[i]{\trans{\vec{v_i}}\vec{z_t}} \right) \nonumber
            \\
        \text{subject to} && \trans{\vec{V}}\vec{V}=\vec{I}
    \label{eqn:ica:6}
\end{eqnarray}

% f}}}

\subsection{扩展ICA到ISA}
% f{{{
从\eqnref{ica:2}得到的特征$\vec{s}$本质上是原始输入$\vec{x}$的线性变换。
线性变换的一个不足便是无法表达不变性,即输入发生任何改变,输出也都会对应变化,
无法对某些变换(如图像的小范围平移、旋转等)保持结果的稳定。

因此对ICA进行如下改进得到ISA:将特征$\vec{s}$分为$K$组,
第$k$组对应的分量下标集合记作$S(k)$。把每组特征看作一个子空间,
将其能量作为该子空间的特征输出。具体而言:
\begin{eqnarray}
    e_k = \sqrt{\sum_{i\in S(k)} s_i^2}
    \label{eqn:isa:0}
\end{eqnarray}

$\vec{e} = \trans{(e_1,\cdots,e_K)}$就是ISA最终的特征输出。

在ISA中,对$\vec{s}$各分量的独立性不做假设,而是假设$\vec{e}$的分量间独立,
并且仍然要求$\trans{\vec{V}}\vec{V}=\vec{I}$以得到尽量丰富的特征。
类似\eqnref{ica:6},基于最大似然对$\vec{V}$求解:
\begin{eqnarray}
    \vec{V}^* &=& \argmax_{\vec{V}}
            \prod_{t=1}^T \left(
            \prod_{k=1}^K\prob[k]{e_k} \right) \nonumber \\
        &=& \argmax_{\vec{V}}
            \prod_{t=1}^T \left(
            \prod_{k=1}^K\prob[k]{\sqrt{
                \sum_{i\in S(k)}\left(\trans{\vec{v_i}}\vec{z_t}\right)^2}
            }\right) \nonumber \\
        \text{subject to} && \trans{\vec{V}}\vec{V}=\vec{I}
    \label{eqn:isa:1}
\end{eqnarray}

一般而言,$p_k(s)$取为拉普拉斯分布(Laplace distribution),
即$p_k(s) = \frac{1}{2b}\exp(-\frac{\abs{s}}{b})$,
并在优化目标上取对数,最终ISA参数求解的形式如下:
\begin{eqnarray}
    L(V) &=& \sum_{t=1}^T \sum_{k=1}^K
        \sqrt{\sum_{i\in S(k)}\left(\trans{\vec{v_i}}\vec{z_t}\right)^2}
        \nonumber \\
    \vec{V}^* &=& \argmin_{\vec{V}} L(V) \nonumber \\
        \text{subject to} && \trans{\vec{V}}\vec{V}=\vec{I}
    \label{eqn:isa:opt}
\end{eqnarray}

而使用ISA提取特征的方法如下:
\begin{eqnarray}
    e_k &=&  \sqrt{\sum_{i\in S(k)} s_i^2} \nonumber \\
    \vec{s} &=& \vec{V}\vec{P}(\vec{x} - \bar{\vec{x}})
    \label{eqn:isa:extract}
\end{eqnarray}

% f}}}

\section{层叠卷积ISA\label{sec:isa:stacked-convolutional}}
% f{{{
以3D的CT扫描图像为例,要将ISA应用于图像处理,
一般是在扫描结果的数据里随机抽取$T$个$p\times p \times p$的图像小块,
把每个图像块平整化,看作一个$p^3$维的向量,将这些向量带入\eqnref{isa:opt}
求解参数$\vec{V}$。

处理图像数据时,ISA可以被放入卷积神经网络的框架中。
在\eqnref{isa:extract}中,
记$\vec{A} = \vec{V}\vec{P}$,$\vec{b}=-\vec{A}\bar{\vec{x}}$,
则$\vec{s}=\vec{A}\vec{x} + \vec{b}$可以看作一个全连接隐层输出。
记$\vec{A}$的维度是$m\times n$,而当$n=p^3$、
$\vec{x}$对应于一个$p\times p \times p$的图像块时,
$\vec{A}$可对应看作一个$m\times 1 \times p \times p \times p$的卷积核,
而其后的子空间的能量响应则可以看作是在多通道3D图像上进行的跨通道的非线性操作,
从而单层ISA可以看作由卷积以及非线性操作组成的卷积神经网络,
如\figref{isa:nn}所示。
在这种框架下,单层的ISA可以被用于任意大小的输入图像上,
在图像上的每个点密集提取特征;在GPU的辅助下可以达到很高的速度。

\begin{figure}[h!]
    \addplot{res/placeholder.png}
    \caption{单层ISA对应的卷积神经网络结构}
    \label{fig:isa:nn}
\end{figure}

但上述标准的单层ISA有两大缺点:
\begin{enumerate}
    \item 只有一层非线性,无法表达更高层更复杂的结构特征
    \item 由于优化过程中需要不断将权重矩阵正规化为正交阵,
        该过程复杂度为$\Theta(n^3)$,因此当输入图像块的维度较大时,
        训练会非常耗时。
\end{enumerate}

为了解决这两个问题,可以将多个这样的卷积ISA层叠起来,
从而作为一个深度特征提取器来使用。训练时可采取贪心逐层训练的方法,
在训练底层时使用较小的图像块,随后转换为卷积形式,
用较大的图像块作为输入并提取特征,在提取出的特征上再训练下一层ISA。

% f}}}

\section{训练方法及其实现}
% f{{{
为了优化\eqnref{isa:opt},我们使用带投影的整批梯度下降的方法。
先从均匀分布中随机采样得到初始权重矩阵$\vec{V_0}$;随后按如下规则迭代更新:
\begin{eqnarray}
    \vec{W_i} &=& \vec{V_{i-1}} - \alpha \frac{\partial L}{\partial
        \vec{V}}(\vec{V_{i-1}}) \nonumber  \\
        \vec{V_i} &=& \left(
            \vec{W_i}\trans{\vec{W_i}}\right)^{-\frac{1}{2}}\vec{W_i}
    \label{eqn:isa:train}
\end{eqnarray}
其中$\vec{W_i}$是沿梯度方向更新后的权重矩阵,而$\vec{V_i}$则是将$\vec{W_i}$
正规化使其成为正交阵。$\alpha$为学习速率,需要调整到合适的值使得$L(\vec{V_i})$
能较快下降而又不至于发生不稳定震荡。训练直到$L(\vec{V_i})$收敛才停止。

在实现方面,我们使用了theano\cite{bergstra+al:2010-scipy}作为训练框架,
用python实现训练功能。theano是一个符号计算框架,支持自动求导,
可以透明地实现GPU和CPU计算后端切换,同时内置了矩阵乘法、
卷积等各种常见操作的高效实现。

为了加速训练,我们实现了数据并行,可同时利用同一台主机上的多个GPU一起计算。
观察\eqnref{isa:opt}的损失函数$L(V)$,及\eqnref{isa:train}的更新规则,
可以发现在计算$\vec{W_i}$时很容易进行数据并行,
只需要将$T$个训练样本拆分成$N$份,各自求出的梯度相加后即可得到总体梯度,
然后再在CPU上更新$\vec{W_i}$及正规化得到$\vec{V_i}$。
这样单层的实际训练时间可以在半小时以内。


\subsection{对实现的简单验证}
在本小节中,我们通过简单的合成数据,来对ISA实现的正确性进行验证。
考虑各维服从独立高斯分布的40维随机变量$\vec{x}=\trans{(x_1,\cdots,x_{40})}$,
和各维服从独立均匀分布的10维随机变量$\vec{y}=\trans{(y_1, \cdots, y_{10})}$,
令
\begin{eqnarray}
    \vec{z} &=& \trans{(z_1, \cdots, z_{40})} \nonumber \\
    z_{4i+j} &=& x_{4i+j+1}y_{i+1} \nonumber \\
    && \forall 0 \le i \le 9,\,0 \le j \le 3
\end{eqnarray}

这样定义$\vec{z}$有两大好处,
一方面$\vec{z}$满足了ISA所需要的super-Gaussian分布,
另一方面将$\vec{z}$的各分量每4个分一组,则组内分量间也有了依赖性,
可以测试ISA的性能。采样$10000$个$\vec{z}$得到$40\times 10000$的矩阵$\vec{Z}$,
然后再从均匀分布中采样得到一个$64\times 40$的矩阵$\vec{M}$作为混合矩阵,
用$\vec{I} = \vec{M}\vec{Z}$作为最终呈现给ISA算法的输入。
为了评价ISA的效果,在求得ISA的权重矩阵$\vec{V}$后,
带入\eqnref{ica:2}中,然后计算$\corr{s_i^2, s_j^2}$作为$(i, j)$
处的元素绘制在\figref{isa:test}中,可以看出ISA能还原出这些随机变量间的内在结构,
被分在同一组的变量间也有较高的相关性。

\begin{figure}[h!]
    \addplot{res/isa-toyeg.eps}
    \caption{在合成数据上ISA还原出的输入变量的内在关系}
    \label{fig:isa:test}
\end{figure}

% f}}}

\section{实验配置\label{sec:isa:expr}}
% f{{{
在本文中,均使用两层卷积ISA。第一层训练时输入的图像块大小为
$13\times 13 \times 13$,先用PCA把数据降维到$600$维,
每个子空间包含$2$个线性特征,输出维度为$300$维;
第二层的原始输入图像块大小为$21 \times 21 \times 21$,
先在图像块上以$8$为步长、用第一层卷积ISA得到$300$通道的$2\times 2 \times 2$
的图像块,共$2400$维作为第二层ISA训练的输入向量,PCA降维至$200$维,
子空间大小为$4$,最终输出特征维度为$50$维。

在SLIVER07的腹腔CT扫描数据上习得的第一层特征检测器如\figref{isa:filter}所示。
观察可以发现,其中很多都像检测各种朝向边缘的Gabor filter;
相邻两个检测器属于同一个子空间,可以看到它们的形态相似而在相位上有区别,
这也表明了一个子空间所对应的特征具有一定的平移不变性。

\begin{figure}[h!]
    {
        \addplot{res/isa-filter.png}
        \caption{在实际数据上用ISA习得的第一层特征检测器}
        \label{fig:isa:filter}
    }
    \footnotesize
    图中展示了每个输出通道所对应的检测器,由于检测器本身是3D的,
    这里为了展示方便,仅选取了其中某一维的中间面片。
\end{figure}

% f}}}

\section{小结与讨论\label{sec:ISA:discuss}}
% f{{{
本节主要对ICA和ISA进行了介绍,描述了ISA到卷积神经网络的转化,并简述了训练方法。

ISA作为一种非监督学习的方法,其习得的特征可以自发的展现一定的平移和相位不变性
\cite{hyvarinen2000emergence}。
层叠卷积ISA被应用在了视频中的动作识别\cite{le2011learning}、
人脑MRI扫描图像的配准\cite{wu2013unsupervised}等领域,均取得不错的结果。

然而,叠卷积ISA也有一定的局限性:
\begin{enumerate}
    \item 卷积核大小需要与输入图像块的大小相同,使得对应的卷积核较大,
        而较大的卷积核会导致较慢的运行时间与较多的内存占用;
    \item 训练时采取贪心逐层训练的方法,缺乏全局优化的过程;
    \item 层数少,无法表达更复杂的结构,
        而且单层ISA的训练也要求输入向量维度不能太少,
        因此要想构造深层网络就需要很大的输入图像块;
    \item 最主要的一个缺陷是,ISA是一种完全非监督的方法;
        而实际应用中,我们往往希望特征对一定的平移、旋转等扰动具有不变性,
        这种要求无法整合进ISA的框架。
        虽然在实验中人们发现ISA习得的特征具有一定平移和相位不变性,
        可是对这种不变性的形式和程度都没有理论保障。
\end{enumerate}
% f}}}

% vim: filetype=tex foldmethod=marker foldmarker=f{{{,f}}}

% $File: CNN.tex
% $Date: Sun May 31 21:18:00 2015 +0800
% $Author: jiakai <jia.kai66@gmail.com>

\chapter{基于深度卷积神经网络的特征提取}
本章主要介绍基于深度卷积神经网络的特征提取方法。基于不同的损失函数,
本章提出两个模型,
但其基本思想都是针对\secref{ISA:discuss}提到的ISA方法的局限性,
通过人工构造的损失函数,引导网络学习出对仿射变换和Gamma校正有鲁棒性、
同时具有较强区分性的特征。

\section{网络结构}

\section{损失函数}
\subsection{分类输出:Softmax与交叉熵损失函数}
\subsection{特征输出:度量学习}

\section{数据增广}
简要形式化描述希望特征对变换$F(A, \gamma)=(affine(A), gamma(\gamma))$
满足的不变性

\subsection{Gamma校正}

\subsection{三维仿射变换}

\section{训练方法}

% vim: filetype=tex foldmethod=marker foldmarker=f{{{,f}}}


% $File: expr.tex
% $Date: Sun Jun 07 23:50:02 2015 +0800
% $Author: jiakai <jia.kai66@gmail.com>

\chapter{评测方法与实验结果}
本章首先提出一种评测方法,其完全基于肝脏分割标注,
无需基于其它的柔性匹配算法,也无需显式计算两个标注间的点对应关系,
可以直接评价特征的优劣,简便高效。随后本章介绍具体实验配置,
并报告基于该评测方法的实验结果。

\section{评测方法}
在实际应用中,往往难以有单一可靠的方法直接评判特征优劣,
而是需要有一个依赖某特征的具体任务,
并通过该特征在该任务上的表现来间接反应特征优劣。
例如,SIFT特征最早被用于物体识别\cite{lowe1999object},
层叠卷积ISA被用于动作识别\cite{le2011learning}、
大脑MRI扫描的柔性配准\cite{wu2013unsupervised}等。
然而,如果任务过于复杂,
结果往往会受到任务相关的具体算法的影响。
例如在Guorong Wu的工作\cite{wu2013unsupervised}中,
作者的评测过程依赖于第三方软件的预处理,
而且发现换用ISA特征后在Demons算法上的配准性能反而变差了,
虽然作者表示这是由于实验过程带来的一些不公平造成的,
但客观而言在这种复杂的环境下确实更难分离出特征本身在最终性能里的贡献。
因此,在本节中,我们将提出一个新的简单而普适的方法来评测特征性能,
以免结果受到过于复杂的任务相关算法的影响。
简单而言,该方法基于人工标定的器官分割掩膜,在不需要具体点对应关系的情况下,
来评价一个特征在测试图像里寻找参考图像中某个点的准确度。

\subsection{基于器官分割标注的曲面匹配}
我们的评测方法要求数据提供对某个器官的分割标注,
例如在本文中我们使用SLIVER07中的肝脏分割标注。
在本小节中,我们将介绍基于大量带器官分割标注的训练数据,
在单个测试图像上寻找某个曲面的方法。为此,我们先定义单点匹配,
并将其扩展到曲面。

\subsubsection{单点匹配}
基于分割标注,我们对每个点都定义{\bf 边界距离},
并基于边界距离来判断单点是否匹配成功。

我们假设器官分割掩膜以二值3D图像$\vec{M}$的形式提供,
$\vec{M}$中某点值为$1$是表示对应点属于目标器官,为$0$时表示不属于目标器官。
对于每个点$(i, j, k)$,定义
\begin{eqnarray}
    N(i, j, k) &=& \prod_{
        \max(\abs{x}, \abs{y}, \abs{z}) = 1}
        M(i + x, j + y, k + z)
\end{eqnarray}
$N(i, j, k)$表示了与$(i, j, k)$相邻的点中是否有不属于目标器官的点。
于是可定义边界点集为:
\begin{eqnarray}
    \partial M &=& \left\{\,\vec{p} : M(\vec{p}) = 1\,\text{且}\,
        N(\vec{p}) = 0\,\right\}
\end{eqnarray}

把每个点看作无向图的顶点,同时在几何上看作一个单位立方体,
对于有公共顶点或公共边的两点间连一条权值为$1$的边,
于是对任意两点$\vec{p}, \vec{q}$,其间存在最短路,
距离记作$s(\vec{p}, \vec{q})$。
对每个点$\vec{p}$,可定义无符号边界距离$D(\vec{p})$和边界距离$d(\vec{p})$:
\begin{eqnarray}
    D(\vec{p}) &=& \min_{\vec{q} \in \partial M} s(\vec{p}, \vec{q}) \\
    d(\vec{p}) &=& (2M(\vec{p})-1)D(\vec{p})
\end{eqnarray}
在涉及多个图像如$\vec{M_1}$、$\vec{M_2}$时,
我们通过脚注形式来区分各自的边界距离:
$d_{\vec{M_1}}(\vec{p})$、$d_{\vec{M_2}}(\vec{q})$。

对于参考图像$\vec{R}$上的某点$\vec{p}$及其特征$\vec{f(p)}$,
我们在测试图像$\vec{T}$上寻找特征距离最小的点$\vec{q}$,
称$\vec{q}$为$\vec{p}$的匹配点,
如果还有$\abs{d_{\vec{R}}(\vec{p}) - d_{\vec{T}}(\vec{q})} \le \theta$,
则认为匹配成功,其中$\theta$为容忍的距离误差,本文中均取$1$。
为了在特征上快速、精确地寻找匹配点,即特征空间上的最近邻,
我们把两两特征间的距离计算转换成矩阵乘法并在GPU上运行。

\subsubsection{曲面匹配\label{sec:expr:match}}
上述单点匹配的判别方法,易受各种随机因素的影响,
在这里我们将其扩展到曲面匹配以提高鲁棒性。

首先根据边界距离,对器官分割标注$\vec{M}$定义参考曲面:
\begin{eqnarray}
    \hat{\vec{M}} &=& \left\{ \vec{p} : d_{\vec{M}}(\vec{p}) = d_0 \right\}
\end{eqnarray}
在本文中,取参考曲面为边界稍靠内的曲面,取$d_0=2$,
这样的一个好处是所有可匹配点的距离在$[1, 3]$间,也都在目标器官内部。

假设我们有$N$个训练数据$\vec{M_1},\cdots,\vec{M_N}$,
我们在$\hat{\vec{M_1}},\cdots,\hat{\vec{M_N}}$上各均匀选取$T$个点,
并在测试图像的上寻找这$NT$个点的匹配点。
为了防止特征只注重了很明显的局部特点而导致匹配点过于集中,
我们把测试图像分成了若干个小方格,每个的大小为$k\times k \times k$,
对于落入同一方格的匹配点,只记录其中距离最小的点,
匹配精确度定义为这些剩下的点中成功匹配的点数占剩下的点总数的比例。
在本文的实验中,均取$T=3000, k=2$。

\figref{expr:match}中给出了在此评测标准下,
达到$63.5\%$准确率的实际匹配点的分布,可以看出总体来说还是符合预期的。

\begin{figure}[h!]
    {
        \addplot{res/expr-match.png}
        \caption{$63.5\%$准确率下实际匹配点的分布}
        \label{fig:expr:match}
    }
    \footnotesize
    该图通过在x轴上按4像素为步长切片绘制。
    其中彩色标注的区域是人工标注的肝脏区域,绿色点为成功匹配点,
    蓝色点为失败匹配点,蓝色、绿色颜色越亮,则表示匹配的特征距离越小。
    为方便查看,每个点用原始点为中心的$5\times 5 \times 5$立方体来表示,
    因此在某些切片上的一些成功匹配点看起来离参考曲面很远,
    其实是来自曲率变化剧烈的区域的其它(未在此绘制的)切片。
\end{figure}

\subsection{ROC曲线及曲线下面积}
基于\secref{expr:match}中针对单个测试图像的曲面匹配方法,
在本小节中我们给出其ROC曲线绘制的方法,以及多个测试图像的整体评分方法。

假设有$M$个测试图像,固定特征距离阈值$\theta$,对每个测试图像,
仅保留$\theta$以下的匹配点,则此时可以得到$(a_1^{(\theta)},
t_1^{(\theta)}),\cdots,(a_M^{(\theta)}, t_M^{(\theta)})$共$M$个二元组,
$a_i^{(\theta)}$表示所有训练图像在第$i$个测试图像上
距离不超过$\theta$的匹配点中成功匹配的比例,即$\theta$限制下的匹配精确度;
$t_i^{(\theta)}$表示这些匹配点占所有训练图像的参考曲面上选中的点的比例,
沿用上节记号,则这些选中的点的总数为$NT$,
$t_i^{(\theta)}$就是$\theta$限制下总匹配点数与$NT$之商。
记$a^{(\theta)}$、$s^{(\theta)}$分别为
$a_1^{(\theta)},\cdots,a_M^{(\theta)}$的均值和方差,
$t^{(\theta)}$为$t_1^{(\theta)},\cdots,t_M^{(\theta)}$的均值,
显然$t^{(\theta)}$随着$\theta$增加是单调不下降的。
遍历$\theta$取值,
将所有$(t^{(\theta)}, a^{(\theta)})$对应的平面点连接起来,
就得到了ROC曲线,
同时可以作出$a^{(\theta)} \pm s^{(\theta)}$的对应区域来反应评测的准确度。


\section{实验配置}

\subsection{实验环境}

\subsection{训练数据}

\subsection{模型参数设定}


\section{实验结果}
曲线,表格


% vim: filetype=tex foldmethod=marker foldmarker=f{{{,f}}}



% $File: discuss.tex
% $Date: Mon Jun 22 17:19:40 2015 +0800
% $Author: jiakai <jia.kai66@gmail.com>

\chapter{总结与展望\label{chap:discuss}}

\section{本文工作总结}
在本文中,我们对基于非监督学习的医学影像特征提取方法进行了研究。

我们以层叠卷积ISA作为基准方法,
在\chapref{ISA}中介绍了其基本数学原理和我们数据并行实现的策略,
同时将相关代码开源。

随后,针对ISA方法固有的缺陷,
我们在\chapref{CNN}中提出需要人工引导特征提取器学习我们想要的不变性、
并让特征同时具有鲁棒性和区分度的总体思想。
为此,我们设计了多分类输出和度量学习两种损失函数,
并将其连接在我们设计的深度卷积神经网络之后。
我们将所需的不变性具体局限在对仿射变换和Gamma校正的不变性,
并针对3D扫描影像提出了对变换函数的参数进行均匀采样的方法。

在\chapref{expr}中,为了测试各种特征提取器的性能,我们设计了一种新的评测标准,
其基本思想是在人工标定的特定器官的分割掩膜上,计算出可以精确定义的边界,
并通过计算训练图像上的边界点在测试图像上的匹配精度来反映特征性能;
这种方法的优点是不需要涉及具体的医学影像处理任务,
避免了其它更复杂的方法影响对特征性能的评测,
同时也只依赖于器官分割标注而无需具体的点对应标注,操作简便。

在我们设计的评测标准上,我们比较了各种算法在一些参数组合下的性能,
并在\secref{expr:discuss}中给出了设计高性能特征提取器所需技巧的一些初步结论。

\section{未来工作展望}
本文对在医学影像上基于非监督学习的特征提取方法进行了初步研究,
然而还是有很多问题没有研究透彻,对于该领域未来的工作,可以进一步探索以下方面:

\paragraph{不同数据上的普适性}
本文中我们仅仅以SLIVER07的腹腔CT扫描数据为例进行了研究,
并未探索我们的结论在MRI等其它来源的数据、
或者其它人体部位的扫描数据、
或者对一些物体(如行李安检)的扫描数据上是否仍然成立。
另一方面,我们也不知道在某个数据集上训练的特征提取器,
在别的数据集上是否也可以直接取得好效果。

\paragraph{网络设计与训练}
由于项目时间和资源有限,我们没有在深度卷积神经网络的设计上进行优化尝试,
而是直接采取了按我们直觉选择的初始设计。网络层数、卷积核大小、通道数、
网络拓扑结构、激活函数的选择、输入大小、特征维度
等各种因素都可能对最终性能造成影响。训练时超参数的调节、
数据增广的程度、网络蒸馏\cite{hinton2015distilling}、
度量学习时的难样本挖掘等训练方法,
也都可能极大地改变最终结果。这方面的最优选择值得进一步探索。

\paragraph{与其它方法的比较}
本文中我们仅仅对深度卷积神经网络和层叠卷积ISA进行了比较,
在未来的工作中也应该与受限玻尔兹曼机、自动编码器、
3D-SIFT等其它方法进行比较,才能对特征提取有更全面深入的理解。

\paragraph{评测标准的可靠性}
在本文中我们提出了新的评测标准,并通过人工观察数据对其有效性进行了大致确认,
但这方面需要更系统的研究。一种可能的研究思路是,
把各种特征接入分割、匹配的各种后期任务,对其在各种任务上的表现进行统计,
并对统计结果和本评测标准给出的排名间的相关性进行考察。

% vim: filetype=tex foldmethod=marker foldmarker=f{{{,f}}}




%%% 其它部分
\backmatter

% 本科生要这几个索引,研究生不要。选择性留下。
\makeatletter
\ifthu@bachelor
  % 插图索引
  \listoffigures
  % 表格索引
  \listoftables
  % 公式索引
  %\listofequations
\fi
\makeatother


% 参考文献
% 注意至少需要引用一篇参考文献,否则下面两行可能引起编译错误。
% 如果不需要参考文献,请将下面两行删除或注释掉。
\bibliographystyle{thubib}
\bibliography{refs}


% 致谢
%%% Local Variables:
%%% mode: latex
%%% TeX-master: "../main"
%%% End:

\begin{ack}
    首先感谢宋亦旭老师在毕设过程中对我的悉心指导。从开题之初对方法的细致讨论,
    到整个毕设过程中对想法、数据、实验的不断交流和讨论,
    直到最后宋老师帮我细致地修改论文,在整个毕设过程中,他给予了我巨大的帮助。
    除了学术内容上的具体帮助,宋老师严谨勤奋的学术态度更是让我获益匪浅:
    宋老师经常很早就会到实验室与我讨论和交流最新想法,对我答辩过程中的用词和幻灯
    片设计细节进行纠正,甚至在最后检查论文阶段宋老师还发现我参考文献中的一处大小
    写错误。宋老师的这些优秀品质让我很受感染,希望以后也能将它们带入工作之中。

    另外要感谢我所在的实习公司Megvii所提供的GPU实验环境,
    才能让这么多实验得以完成;
    感谢曹志敏师兄在一些数学问题上给我的帮助;感谢实验室的乔天学长在环境配置、
    数据准备等方面给予的帮助;感谢张楚悦同学提供的医学影像学方面的资料。
    还要感谢宋琦同学在整个过程中对我的支持和鼓励。

    最后,感谢开源社区,感谢linux, python, theano, numpy,
    opencv, 3DSlicer等各种工具,有了它们才让这些实验成为可能。
    感谢\XeLaTeX 和 \thuthesis, 它们的存在让我的论文写作轻松自在了许多。
\end{ack}


% 附录
\begin{appendix}
% $File: appendix01.tex
% $Date: Wed Jun 10 19:03:45 2015 +0800
% $Author: jiakai <jia.kai66@gmail.com>

\chapter{外文资料翻译}

\section{翻译}
最后再写吧

\section{原文}
\includepdf[pages={-},scale=.9,pagecommand={}]{res/trans-english.pdf}


% vim: filetype=tex foldmethod=marker foldmarker=f{{{,f}}}

\end{appendix}

% 个人简历
%\include{data/resume}

\end{document}

% vim: filetype=tex foldmethod=marker foldmarker=f{{{,f}}}
