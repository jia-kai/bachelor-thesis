% $File: intro.tex
% $Date: Fri May 29 01:27:37 2015 +0800
% $Author: jiakai <jia.kai66@gmail.com>

\chapter{引言}
本章主要阐述本文要研究的问题,简述其背景,并概述目前相关领域的研究现状。
最后给出本文的结构安排。

\section{研究背景}
对医学影像的获取、分析和处理在现代医疗工作中占有重要地位。
常用的基于放射的医学成像技术包括计算机体层成像(Computed Tomography, CT)、
磁共振成像(Magnetic Resonance Imaging, MRI)等。在医学影像技术的辅助下,
人们可以非侵入式的直接观察到患者体内的情况,
对疾病的诊治和人体科学的研究都有重大意义。

对医学影像的分析和处理也是计算机科学相关领域长久以来研究的热点。
但由于其本身的难度以及对结果精确度的极高要求,这方面的自动化方法远不成熟,
只能作为人工辅助,临床上也一般由训练有素的医生来完成对影像的分析工作。
一般而言,医学影像的如下特点使得对其的自动化处理较为困难:
\begin{inparaenum}[\itshape 1\upshape)]
    \item 分辨率不高,如CT的图像矩阵往往只有$512\times 512$\cite{medimging2};
    \item 来源单一而封闭,需要大型扫描设备,而且涉及患者隐私;
    \item 由于人体构造、扫描时体位等不同而呈现很大差异性,
        病变部位则更是千差万别;
    \item 维度高,一般为3D体数据,或者带时间信息的4D图像。
\end{inparaenum}

近年来随着互联网的高速发展带来的海量数据,
以及GPU技术的成熟和普及带来的高密度高速计算能力,
使得深度学习逐渐成为一种实用的高性能通用模型,并在图像分类\cite{he2015delving}、
人脸识别\cite{schroff2015facenet}等领域的特定数据集上超过了人类水平。
但目前将深度学习用于医学影像处理的工作并不多。因此,
本文将尝试使用基于非监督学习的深度学习算法,自动从CT扫描数据中提取特征,
并基于本文提出的评测方法测试各种特征的性能。


\section{深度学习}


\section{医学影像处理}
医学影像的特点、研究的意义、难点,深度学习与此的结合。

并简述肝脏分割的现有方法,引出本文以刚脏分割作为 example task。

\section{论文结构}


% vim: filetype=tex foldmethod=marker foldmarker=f{{{,f}}}
