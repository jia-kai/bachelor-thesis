% $File: CNN.tex
% $Date: Mon Jun 01 23:50:58 2015 +0800
% $Author: jiakai <jia.kai66@gmail.com>

\chapter{基于深度卷积神经网络的特征提取}
本章主要介绍基于深度卷积神经网络的特征提取方法。基于不同的损失函数,
本章提出两个模型,
但其基本思想都是针对\secref{ISA:discuss}提到的ISA方法的局限性,
通过人工构造的损失函数,引导网络学习出对仿射变换和Gamma校正有鲁棒性、
同时具有较强区分性的特征。

\section{网络结构}
深度卷积神经网络一般由卷积、池化、全连接等层组成,
卷积和全连接层后一般接一个非线性函数。在本小节中,
先对这三种层对应的数学操作进行介绍,随后再给出本文所用的网络结构。

\subsection{基本数学操作}


\subsection{本文所用网络结构}
本文仅试验了一种网络结构,其计算层由三层卷积、一层池化、两层全连接组成。
为了与\secref{isa:expr}所描述的ISA模型公平比较,
输入与之相同用$21\times 21 \times 21$的图像块,输出同样为$50$维特征。
图像块在进入网络之前,
先要经过线性变换$\vec{y}=k\vec{x}+b$,其中$k$、$b$为在训练集上求得的标量常数,
使得$E[\vec{y}]=0$,$\max(\abs{\vec{y}})=1$。
网络中所有卷积和全连接层均使用使用ReLU作为非线性,即$f(x)=\max(x, 0)$。
具体的网络结构参数见\tabref{cnn:arch}。

\begin{table}[h!]
    \begin{center}
        \caption{本文所用的深度卷积神经网络结构}
        \label{tab:cnn:arch}
        \begin{tabular}{c|c|c}
            \hline
            {\heiti 层编号} & {\heiti 输入大小} & {\heiti 层内容} \\ \hline
            0 & $1\times 21 \times 21  \times 21$ & 线性变换:$y=kx+b$ \\
            1 & $1\times 21 \times 21  \times 21$ & $Conv(20, 4)$ \\
            2 & $20\times 18 \times 18  \times 18$ & $MeanPooling(2)$ \\
            3 & $20\times 9 \times 9 \times 9$ & $Conv(24, 4)$ \\
            4 & $24\times 6 \times 6 \times 6$ & $Conv(28, 4)$ \\
            5 & $28\times 3 \times 3 \times 3$ & $FC(60)$ \\
            6 & $60$ & $FC(50)$ \\
            7 & $50$ & 损失函数 \\
        \end{tabular}
    \end{center}
    \footnotesize
    注:
    \begin{center}
        \begin{itemize}
            \item $Conv(c, s)$ 表示输出通道数为$c$,
                卷积核大小为$s\times s \times s$,
                步长为$1$的3D卷积
            \item $MeanPooling(s)$ 表示大小为$s\times s \times s$
                且无重叠的均值池化
            \item $FC(n)$ 表示输出维度为$n$的全连接层
        \end{itemize}
    \end{center}
\end{table}

\section{损失函数}
\subsection{分类输出:Softmax与交叉熵损失函数}
\subsection{特征输出:度量学习}

\section{数据增广}
简要形式化描述希望特征对变换$F(A, \gamma)=(affine(A), gamma(\gamma))$
满足的不变性

\subsection{Gamma校正}

\subsection{三维仿射变换}

\section{训练方法}

% vim: filetype=tex foldmethod=marker foldmarker=f{{{,f}}}

