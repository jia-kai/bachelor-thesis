% $File: nasmia.tex
% $Date: Thu Mar 12 00:00:41 2015 +0800
% $Author: jiakai <jia.kai66@gmail.com>

\documentclass {beamer}
%\usetheme {JuanLesPins}
\usetheme{Malmoe}
%\usecolortheme{beaver}
\usepackage{fontspec,amsmath,amssymb,verbatim,mathtools,tikz}

% chinese
\usepackage{zhspacing}
\zhspacing
%\usepackage[noindent,UTF8]{ctexcap}

% biblatex
\usepackage{biblatex}
\bibliography{refs.bib}
\defbibheading{bibliography}{\section{}}
\setbeamertemplate{bibliography item}{\insertbiblabel}
\renewcommand*{\bibfont}{\footnotesize}

% minted
\usepackage{minted}
\newcommand{\inputmintedConfigured}[3][]{\inputminted[fontsize=\small,
	label=#3,linenos,frame=lines,framesep=0.8em,tabsize=4,#1]{#2}{#3}}
% \*src[additional minted options]{file path}
\newcommand{\cppsrc}[2][]{\inputmintedConfigured[#1]{cpp}{#2}}

% math opr
\DeclareMathOperator*{\argmin}{arg\,min}
\newcommand{\trans}[1]{#1^\intercal}

% others
\newcommand{\addgraph}[2]{\begin{center}
\includegraphics[width=#1\paperwidth]{#2}\end{center}}


\title{基于深度学习的医学影像柔性配准算法研究}
\author {贾开}
\institute {清华大学}
\date{\today}

\begin{document}
\frame[plain]{\titlepage
    \begin{center}
        \begin{tabular}{ll}
            指导老师 & 宋亦旭 \\
            报告人 &  贾开 \\
            学号 & 2011011275
        \end{tabular}
    \end{center}
}

\AtBeginSection[]
{
    \begin{frame}<beamer>
        \frametitle{目录}
        \tableofcontents[currentsection]
    \end{frame}
}


\begin{frame}{目录}
    \tableofcontents
\end{frame}

\section{选题背景和意义}
% f{{{

\subsection{医学影像}

\begin{frame}{医学影像的自动分析和处理}
    \begin{itemize}
        \item 长期以来的研究热点,驱动机器视觉发展,
            但尚未完全解决
        \item 减轻医生负担,提高诊断准确率
        \item 交叉学科地带,
            整个领域随着医学、医学影像、计算机科学等领域的发展而不断变化
    \end{itemize}
    \addgraph{0.3}{res/mi-concept.jpg}
    \tiny{image source:
        \url{http://en.wikipedia.org/wiki/Medical\_imaging}}
\end{frame}


\begin{frame}{医学影像的配准}
    \begin{itemize}
        \item 病例到标准模板的匹配
        \item 术前术后的匹配
        \item 刚性配准(rigid registration) vs \\
            柔性配准(non-rigid
            registration/deformable registration)
    \end{itemize}
    \addgraph{0.5}{res/example-brain.png}
    \tiny{image source: \cite{shen2007image}}
\end{frame}

\begin{frame}{医学影像的特点及其配准的难点}
    医学影像的\alert{特点}:
    \begin{itemize}
        \item 影像质量不高:分辨率、信噪比等都比较低
        \item 来源单一而封闭:需大型扫描设备,涉及患者隐私问题
        \item 正常部位一致性高,病变部位难以有稳定模型来描述
        \item 一般为3D空间灰度图像,或带时间信息的4D图像
    \end{itemize}
    \pause
    医学影像配准的\alert{难点}:
    \begin{itemize}
        \item 非适定性问题(Hadamard, 解存在、唯一、随输入连续变化)
        \item 需要处理噪声、低分辨率、病变部位的高度不一致等问题
    \end{itemize}
\end{frame}

\begin{frame}{柔性配准方法概述}
    主要包含三个部分:
    \begin{itemize}
        \item 柔性变形模型
        \item \alert<2>{相似度量函数:根据关键点附近的特征值衡量图像相似性}
        \item 优化方法
    \end{itemize}
    \visible<2>{
        本研究尝试用更好的特征实现更好的相似度量函数,
        从而提高配准精度
    }
\end{frame}

\subsection{深度学习}

\begin{frame}{更好的特征}
    \begin{description}
        \item[传统特征] \hfill \\
            SIFT, HOG, SURF, LBP, \dots \\
            基于人的知识,\alert{人工设计} \\
        \item<2>[深度学习] \hfill \\
            从数据中自动发掘特征基于数据客观分布,\alert{自动发现}
    \end{description}
\end{frame}

% f}}}


\section{相关领域研究现状}
% f{{{

\subsection{柔性配准}

\begin{frame}{问题描述}
    柔性配准的问题可形式化表达如下\cite{sotiras2013deformable}:
    \begin{eqnarray*}
        W^* = \argmin_{W} M(T, S \circ W) + R(W)
    \end{eqnarray*}
    其中$W$为待求变换的参数表示,$S$和$T$分别是源影像与目标影像,
    $M(A, B)$评价$A$和$B$的不相似度,
    $R(W)$为$W$上基于先验知识的正则项。
    \begin{description}
        \item[柔性变形模型] $\circ W$
        \item[相似度量函数] $M$
    \end{description}
\end{frame}

\begin{frame}{拟用方法}
    本研究中拟采取Dinggang Shen提出的HAMMER及其改进方法
    \cite{shen2002hammer,shen2007image}
    \footnotesize
    \begin{eqnarray*}
        h(u) &=& u + d(u) \\
        E &=& \sum_{u}
        \omega_T(u)\left(\frac{
            \sum_{z\in n(u)}\varepsilon(z)(1-m(a_T(z), a_S(h(z))))}{
                \sum_{z\in n(u)}\varepsilon(z)}\right) \\
        & & + \sum_{v}\omega_S(v)\left(\frac{
            \sum_{z\in n(v)}\varepsilon(z)(1-m(a_T(h^{-1}(z)), a_S(z)))}{
                \sum_{z\in n(v)}\varepsilon(z)}\right) \\
        & & + \beta\sum_u\Vert\nabla^2d(u)\Vert
    \end{eqnarray*}
\end{frame}


\subsection{深度学习}

\begin{frame}{深度学习:沉默后的爆发}
    \begin{itemize}
        \item 深度学习:人工神经网络的时髦别名\\
            相关理论已在二十多年前成型,但当时并未展示出明显优势
        \item LSVRC2012中,AlexNet取得前5输出的15\%错误率,
            而最好的传统方法为26\%
        \item 深度学习已在机器视觉、机器翻译、自然语言处理、
            语音识别、文本识别等很多领域超越或远超传统方法
        \item 目前尝试将其用在医学影像特征提取的工作不多
    \end{itemize}

    \addgraph{0.5}{res/alexnet.png}
    \tiny{image source: \cite{krizhevsky2012imagenet}}
\end{frame}

\begin{frame}{深度学习广泛应用的基础}
    \begin{description}
        \item[理论可行性] 足够大的神经网络可以无限逼近任意连续实函数
        \item[大数据] 相对二十年前,现在有了足够多的数据,使得大模型不易过拟合
        \item[训练技巧] 初始化、非线性、优化等各个环节的tricks
        \item[硬件设备] GPU使得快速大规模浮点运算成为可能
    \end{description}
\end{frame}

\begin{frame}{深度学习的关键因素}
    \begin{description}
        \item[模型] 卷积神经网络(CNN),递归神经网络(RNN),受限玻耳兹曼机(RBM),
            深度信念网络(DBN)等
        \item[优化方法] 随机梯度下降及其变种(Momentum, AdaGrad, rmsprop等)
        \item[tricks] Pretrain, Drop Out, ReLU, Maxout,
            Max/Average Pooling, Batch Normalization等等
    \end{description}
\end{frame}

\begin{frame}{非监督的深度学习}
    \begin{description}
        \item[降噪自动编码器\cite{vincent2010stacked}]
            \hfill \\
            $h=\sigma(Wx+b), \tilde{x}=\sigma(\trans{W}h+c)$
        \item[层叠卷积自动编码器\cite{masci2011stacked}]
            \hfill \\
            $h=\sigma(x*W+b), \tilde{x}=\sigma(h*\tilde{W}+c)$
        \item[基于数据增广和区分式分类训练\cite{dosovitskiy2014discriminative}]
            \hfill \\
            采图像块并进行变换,要求从同一个图像块变换得到的图像块被分到一类
    \end{description}
\end{frame}

% f}}}

\section{本研究的内容计划}

\subsection{研究内容概览}

\begin{frame}{研究概览}
    \begin{description}
        \item[研究目标] \hfill \\
            比较几种深度学习方法,及其所习得特征用于医学影像配准的性能
        \item[研究内容] \hfill \\
            \begin{itemize}
                \item 实现\cite{wu2013unsupervised}中算法,
                    并在ADNI数据集上训练和测试
                \item 尝试用后述的其他深度学习方法习得特征,并比较效果
                \item 扩展内容1:在肝脏影像上使用上述方法进行特征学习和配准
                \item 扩展内容2:将上述特征用于半自动的病变区域选取
            \end{itemize}
        \item[研究重点] \hfill \\
            如何使用深度学习得到更好的特征?
    \end{description}
\end{frame}

\begin{frame}{基线算法:层叠卷积ISA\cite{le2011learning,wu2013unsupervised}}
    \begin{description}
        \item[概述]
            \begin{enumerate}
                \item 在训练数据中采出小的图像块,进行ISA,
                    得到权重矩阵
                \item 将上述权重矩阵重新打乱形成卷积核,在训练数据上卷积,
                    得到新的数据再次进行ISA
                \item 上述两次ISA的权重矩阵可层叠起来作为两次卷积使用,
                    即为得到特征提取器
            \end{enumerate}
        \item<2>[问题]
            \begin{enumerate}
                \item 层数少,作者也未实验更深更大的网络是否能带来性能提升
                \item 贪心地逐层训练,缺少全局优化过程
                \item 基于纯统计的方法学习,
                    训练过程中并未加入关于不变性的先验知识
            \end{enumerate}
    \end{description}
\end{frame}


\subsection{改进部分}

\begin{frame}{改进方法:基础改进}
    针对前述层数少、缺乏全局优化、缺乏先验知识的问题:
    \pause
    \begin{itemize}
        \item<+-> 受目前主流ImageNet分类算法的启发,使用CNN网络结构
        \item<+-> 将3D卷积(宽,高,通道)扩展到4D(长,宽,高,通道)
        \item<+-> 可引入Pooling层,以及更多卷积层
        \item<+-> 先验1:基于所期望的不变性进行数据增广:旋转、平移、拉伸、
            基于随机场进行柔性扭曲等
        \item<+-> 先验2:特征应具有区分性:随机采集图像块,
            对每块增广后作为一类,随后监督地进行分类问题训练
    \end{itemize}
\end{frame}

\begin{frame}{改进方法:细节优化}
    可以尝试的想法:
    \begin{itemize}
        \item 使用自然图片或视频进行预训练,随后在医学影像上调优,
            解决医学影像数据量少的问题
        \item 使用前述的层叠卷积ISA方法得到的权重作为网络初始权重,
            随后进行全局调优,期望基于统计习得的权重优于随机初始权重
        \item 基于\cite{vincent2010stacked}
            的方法加入输入降噪和重建的监督信号
        \item \dots
    \end{itemize}
\end{frame}

\subsection{时间表}
\begin{frame}{时间表}
    \begin{tabular}{ll}
        寒假 & 研究数据格式,导出数据(已基本完成) \\
        1-3周 & 实现卷积ISA,得到基本可用的特征 \\
        4-7周 & 实现、调优并测试基本的配准算法 \\
        8周 & 准备期中检查 \\
        9-13周 & 尝试改进的深度学习方法 \\
        14-15周 & 如果进展顺利,则尝试扩展内容 \\
        16周 & 整理实验内容,完成论文
    \end{tabular}
\end{frame}


\section{ }
\subsection{ }
\frame{\begin{center}\huge{Thanks!}\end{center}}

\section[]{参考文献}
\begin{frame}[allowframebreaks]{参考文献}
    \printbibliography
\end{frame}


\end{document}

% vim: filetype=tex foldmethod=marker foldmarker=f{{{,f}}} 

